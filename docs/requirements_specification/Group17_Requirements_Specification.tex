\documentclass[11pt]{article}
\usepackage{graphicx} % To include graphics 
\usepackage[tbtags]{amsmath} % tbtags to make eq number at last line
\usepackage{amssymb}
%\usepackage{wrapfig} % For figures next to text
\usepackage{booktabs} % Better tables
% \usepackage{physics} 
\usepackage{cases} % For cases with separate labels
\usepackage{hyperref}% For cross-referencing
\hypersetup{colorlinks=true, linkcolor=blue, urlcolor=blue, citecolor=blue}
%\usepackage{pgfplots} % For drawing graphs
\usepackage{parskip} % To remove indentation and set paragraph spacing
\usepackage[margin=1in]{geometry}
%\usepackage{cancel} % To cancel out terms
\usepackage{listings} % To write code
\usepackage{xcolor} % To write color text/math by \textcolor{color}{text}
\usepackage{enumitem} % Better list environment
\usepackage{fancyhdr} % For custom headers and footers
% \usepackage{amsthm} % To write theorems and proofs
% \renewcommand\qedsymbol{$\blacksquare$}
% \usepackage{algorithm}
% \usepackage{algpseudocode}

%%%%%%%%%% Configure fancy headers %%%%%%%%%%
\pagestyle{fancy}

%%%%%%%%%% listings config (for writing code) %%%%%%%%%%
% \definecolor{codegreen}{rgb}{0,0.6,0}
% \definecolor{codegray}{rgb}{0.5,0.5,0.5}
% \definecolor{codepurple}{rgb}{0.58,0,0.82}
% \definecolor{backcolour}{rgb}{0.95,0.95,0.92}

% \lstdefinestyle{mystyle}{
%     frame=tb,
%     basicstyle=\ttfamily,
%     backgroundcolor=\color{backcolour},   
%     % commentstyle=\color{codegreen},
%     % keywordstyle=\color{magenta},
%     % stringstyle=\color{codepurple},
%     breakatwhitespace=false,         
%     breaklines=true,                 
%     captionpos=b,                    
%     keepspaces=true,                 
%     numbers=left,                    
%     numbersep=5pt,                  
%     showspaces=false,                
%     showstringspaces=false,
%     showtabs=false,                  
%     tabsize=4
% }
% \lstset{style=mystyle}

% Title
\title{CSCI3100 Project Requirements Specification}
\author{
    \textbf{Group 17} \\[1em]
    \begin{tabular}{ll}
        Ng Ching Yin & (1155175606) \\
        Lam Hoi Chun & (1155192755) \\
        Lai Wing Fai & () \\
        Zou Zhi Hong & (1155204947)
    \end{tabular}
}
\date{\today}

\begin{document}
\maketitle

% Table of contents
{
    \hypersetup{linkcolor=black}
    \tableofcontents
}

\newpage

% Document Revision History
% Update this only before pull request, don't update every commit
\section{Document Revision History}
\begin{table}[h]
    \centering
    \caption{Document Revision History}
    \begin{tabular}{p{1.5cm} p{6cm} p{2cm} p{3cm}}
        \toprule
        Version & Revised By & Revision Date & Comments \\
        \midrule
        0.0.1 & Ng Ching Yin, Lam Hoi Chun, Zou Zhi Hong & 11 Nov 2025 & Initial release \\
        \bottomrule
    \end{tabular}
    \label{tab:docs_rev_hist}
\end{table}

% Introduction
\section{Introduction}

\subsection{Purpose}

This document defines the user requirements for the new software project management
tool for the Demon Slayers Corps. This document will serve as a basis for both
stakeholders and developers in subsequent development activities.

\subsection{Project Background}

In a realm where demon slayers fought valiantly against the forces of darkness, the boss of
the Demon Slayer Corps struggled with managing their software projects, leading to chaotic
battles and the tragic loss of many brave warriors. Recognizing the dire need for better
organization, Kei, the CSCI3100 lecturer, and his eager students decided to step in, determined
to create a robust software project management tool tailored specifically for the Demon Slayers.

With their combined skills in coding and project management, they envisioned a system that
would streamline communication, track progress, and allocate resources effectively, ultimately
empowering the slayers to focus on their true mission: vanquishing demons and protecting
humanity.

\subsection{Business Objectives}
The high-level business objectives for the project are to:
\begin{enumerate}
    \item reduce inefficiency in task and resource allocation, including
    \begin{itemize}%[leftmargin=*]
        \item[(a)] average time from task creation to assignment;
        \item[(b)] workload variance across team members;
    \end{itemize}
    \item track mission and development progress in real time;
    \item streamline communication among developers;
    \item provide useful tools to facilitate software development;
\end{enumerate}

\subsection{Project Scope}
The new project management tool aims to streamline software team coordination by 
providing task tracking, scheduling, messaging, and an LLM chat interface to accelerate
development workflows.

The major functions to be provided by the tool are listed below:
\begin{enumerate}
    \item Kanban board for task management and progress tracking;
    \item Calendar to track and visualize deadlines;
    \item Discussion system to streamline communication among developers;
    \item LLM chat interface to speed up development progress;
\end{enumerate}

\section{Identified Risks, Assumptions and Constraints}
\subsection{Identified Risks}
\begin{table}[h]
    \centering
    \caption{Identified Risks}
    \begin{tabular}{p{1cm} | p{3.5cm} | p{1.8cm} | p{3cm} | p{1.3cm} | p{2cm}}
        \toprule
        Risk No. 
        & Description 
        & Likelihood 
        & Impact 
        & Rating 
        & Possible Resolutions \\
        
        \midrule
        R1 
        & LLM third-party API failure
        & Medium
        & Slows down \text{development} progress
        & Low 
        & Use \text{multiple} LLM providers; Alert \text{system} to detect \text{failure} \\

        \midrule
        R2
        & LLM abuse; high API cost
        & High
        & Increase API expense significantly
        & High
        & Limit API usage by plan; \\

        \midrule
        R3
        & Local devices / out-of-sync user seeing conflicting information
        & High
        & User confusion and mistrust in the system
        & High
        & Warn user when out-of-sync; \text{Implement} sync \text{recovery} mechanism \\

        \bottomrule
    \end{tabular}
    \label{tab:docs_identified_risks}
\end{table}

\subsection{Assumptions}
\begin{table}[h]
    \centering
    \caption{Assumptions}
    \begin{tabular}{p{2cm} p{3.5cm} p{10cm}}
        \toprule
        Assumption No. 
        & Title 
        & Description \\
        
        \midrule
        A1 
        & Supported OS
        & All popular OS, including Windows, Linux, macOS, iOS, Android and iPadOS \\

        \midrule
        A2
        & Apple minimum OS version 
        & iOS 26, iPadOS 26, and macOS 26  \\

        \bottomrule
    \end{tabular}
    \label{tab:docs_assumptions}
\end{table}

\subsection{Constraints}
\begin{enumerate}
    \item Development Team Expertise: The system shall be developed by
    CSCI3100 students as part of their course project. The design and
    implementation are constrained by the students' current knowledge
    of technologies, databases, and software engineering principles, 
    and must be completed within a single academic semester.
    \item Budgetary Constraints: The project has no monetary budget for
    proprietary software licenses, Application Programming Interface (API)
    calls, or cloud hosting services. The system must be built exclusively
    with free and open-source technologies and frameworks.
    \item LLM Integration Limitation: The integration of the Large Language
    Model (LLM) chat interface is dependent on the availability and usage
    limits of a free-tier or publicly accessible LLM API (e.g., OpenAI GPT,
    Llama). The system cannot guarantee unlimited or high-volume usage without
    incurring costs.
    \item Scope Creep Control: To ensure timely delivery for the semester's end,
    the project scope is fixed upon the agreement of the initial requirements.
    Any proposed additional features will be deferred to a hypothetical "Phase 2" and will
    not be considered for the initial release.
\end{enumerate}

\section{Requirements}

devban is an application that combines gamified to-do lists and daily planners, providing users
with a more engaging and motivating approach to organizing their routines and managing their tasks.
It is designed for users who want to enjoy the benefits of combining effective time management
with sustained motivation. By including gamification elements, the application could help users to
build productive habits, manage the tendency to delay tasks, and gain satisfaction from meeting
personal goals.

\subsection{Software Requirements}
\subsubsection{User Interface}
\begin{itemize}
    \item Calendar Area
    \begin{itemize}
        \item Monthly calendar grid view displaying all days of the current month
        \item Event count indicators showing the number of scheduled events 
        \item Event management features:
        \begin{itemize}
            \item Create new events with title, date, optional start time, and optional end time
            \item Edit existing events to modify their details
            \item Delete events
            \item Toggle completion status of events (mark as completed/incomplete)
            \item Visual indicators for completed events (strikethrough text and checkmark icon)
        \end{itemize}
    \end{itemize}
    
    \item Message Area: "AskLLM"
    \begin{itemize}
        \item Scrollable list of chat bubbles (user vs. LLM) with the newest chats at the bottom
        \item Input area containing a multiline text editor with a Submit button
    \end{itemize}
\end{itemize}

\subsubsection{Operating System and Environment}
This application is developed and deployed for Apple devices running iOS, iPadOS, and macOS, which require iOS 26, iPadOS 26, or macOS Sequoia 26 as per the latest Apple developer policies.

% \appendix
% \addcontentsline{toc}{section}{Appendix}
% \section*{Appendix}
% \section{References}

\end{document}
