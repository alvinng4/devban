\documentclass[11pt]{article}
\usepackage{graphicx} % To include graphics 
\usepackage[tbtags]{amsmath} % tbtags to make eq number at last line
\usepackage{amssymb}
%\usepackage{wrapfig} % For figures next to text
\usepackage{booktabs} % Better tables
% \usepackage{physics} 
\usepackage{cases} % For cases with separate labels
\usepackage{hyperref}% For cross-referencing
\hypersetup{colorlinks=true, linkcolor=blue, urlcolor=blue, citecolor=blue}
%\usepackage{pgfplots} % For drawing graphs
\usepackage{parskip} % To remove indentation and set paragraph spacing
\usepackage[margin=1in]{geometry}
%\usepackage{cancel} % To cancel out terms
\usepackage{listings} % To write code
\usepackage{xcolor} % To write color text/math by \textcolor{color}{text}
\usepackage{enumitem} % Better list environment
\usepackage{fancyhdr} % For custom headers and footers
% \usepackage{amsthm} % To write theorems and proofs
% \renewcommand\qedsymbol{$\blacksquare$}
% \usepackage{algorithm}
% \usepackage{algpseudocode}

%%%%%%%%%% Configure fancy headers %%%%%%%%%%
\pagestyle{fancy}

%%%%%%%%%% listings config (for writing code) %%%%%%%%%%
% \definecolor{codegreen}{rgb}{0,0.6,0}
% \definecolor{codegray}{rgb}{0.5,0.5,0.5}
% \definecolor{codepurple}{rgb}{0.58,0,0.82}
% \definecolor{backcolour}{rgb}{0.95,0.95,0.92}

% \lstdefinestyle{mystyle}{
%     frame=tb,
%     basicstyle=\ttfamily,
%     backgroundcolor=\color{backcolour},   
%     % commentstyle=\color{codegreen},
%     % keywordstyle=\color{magenta},
%     % stringstyle=\color{codepurple},
%     breakatwhitespace=false,         
%     breaklines=true,                 
%     captionpos=b,                    
%     keepspaces=true,                 
%     numbers=left,                    
%     numbersep=5pt,                  
%     showspaces=false,                
%     showstringspaces=false,
%     showtabs=false,                  
%     tabsize=4
% }
% \lstset{style=mystyle}

% Title
\title{CSCI3100 Project Requirements Specification}
\author{
    \textbf{Group 17} \\[1em]
    \begin{tabular}{ll}
        Ng Ching Yin & (1155175606) \\
        Lam Hoi Chun & (1155192755) \\
        Lai Wing Fai & () \\
        Zou Zhihong & (1155204947)
    \end{tabular}
}
\date{\today}

\begin{document}
\maketitle

% Table of contents
{
    \hypersetup{linkcolor=black}
    \tableofcontents
}

\newpage

% Document Revision History
% Update this only before pull request, don't update every commit
\section{Document Revision History}
\begin{table}[h]
    \centering
    \caption{Document Revision History}
    \begin{tabular}{cccc}
        \toprule
        Version & Revised By & Revision Date & Comments \\
        \midrule
        0.0.1 & Ng Ching Yin, Lam Hoi Chun & 24 Oct 2025 & Initial draft \\
        \bottomrule
    \end{tabular}
    \label{tab:docs_rev_hist}
\end{table}

% Introduction
\section{Introduction}

\subsection{Purpose}

This document defines the user requirements for the new software project management
tool for the Demon Slayers Corps. This document will serve as a basis for both
stakeholders and developers in subsequent development activities.

\subsection{Project Background}

In a realm where demon slayers fought valiantly against the forces of darkness, the boss of
the Demon Slayer Corps struggled with managing their software projects, leading to chaotic
battles and the tragic loss of many brave warriors. Recognizing the dire need for better
organization, Kei, the CSCI3100 lecturer, and his eager students decided to step in, determined
to create a robust software project management tool tailored specifically for the Demon Slayers.

With their combined skills in coding and project management, they envisioned a system that
would streamline communication, track progress, and allocate resources effectively, ultimately
empowering the slayers to focus on their true mission: vanquishing demons and protecting
humanity.

\subsection{Business Objectives}
The high-level business objectives for the project are to:
\begin{enumerate}
    \item reduce inefficiency in task and resource allocation, including
    \begin{itemize}%[leftmargin=*]
        \item[(a)] average time from task creation to assignment;
        \item[(b)] workload variance across team members;
    \end{itemize}
    \item track mission and development progress in real time;
    \item streamline communication among developers;
    \item provide useful tools to facilitate software development;
\end{enumerate}

\subsection{Project Scope}
The new project management tool aims to streamline software team coordination by 
providing task tracking, scheduling, messaging, and an LLM chat interface to accelerate
development workflows.

The major functions to be provided by the tool are listed below:
\begin{enumerate}
    \item Kanban board for task management and progress tracking;
    \item Calendar to track and visualize deadlines;
    \item Messaging system to streamline communication among developers;
    \item LLM chat interface to speed up development progress;
\end{enumerate}

% Assumptions and Dependencies
\section{Assumptions and Dependencies}
\begin{table}[h]
    \centering
    \caption{Assumptions and Dependencies}
    \begin{tabular}{p{2cm} p{3.5cm} p{10cm}}
        \toprule
        Assumption No. 
        & Title 
        & Description \\
        
        \midrule
        A1 
        & Supported OS
        & All popular OS, including Windows, Linux, macOS, iOS, Android and iPadOS \\

        \midrule
        A2
        & Apple minimum OS version 
        & iOS 26, iPadOS 26, and macOS 26  \\
 
        \bottomrule
    \end{tabular}
    \label{tab:docs_assumptions_and_dependencies}
\end{table}

\section{Requirements}
\subsection{Completing the Software Development Process}
\subsubsection{Requirements Specification}
devban is an application that combines gamified to-do lists and daily planners, providing users
with a more engaging and motivating approach to organizing their routines and managing their tasks.
It is designed for users who want to enjoy the benefits of combining effective time management
with sustained motivation. By including gamification elements, the application could help users to
build productive habits, manage the tendency to delay tasks, and gain satisfaction from meeting
personal goals.

\subsubsection{Design and Implementation}
devban offers multiple useful features, including customizable to-do lists, calendar-based planning, tagging for task categorization, progress tracking, and reward systems that encourage consistent productivity. Besides, the user interface enables users to view pending tasks, deadlines, and
recurring activities easily. Furthermore, the integrated calendar and note functions enhance the
user’s ability to plan future tasks, reflect on previous achievements, and maintain organized records
of the tasks.

\subsection{Functional Requirements}

\subsubsection{FR1: User Registration and Authentication}
\begin{itemize}
    \item The system shall allow users to create an account, log in, and log out securely.
\end{itemize}

\subsubsection{FR2: Task Management (Kanban Board)}
    Users shall be able to:
\begin{itemize}

  \item create, update, and delete tasks;
  \item assign tasks to team members;
  \item move tasks between “To Do”, “In Progress”, and “Done” columns.
\end{itemize}

\subsubsection{FR3: Messaging System}
\begin{itemize}
    \item Users shall communicate through an integrated real-time chat system.
\end{itemize}

\subsubsection{FR4: LLM Chat Assistant}
The system shall provide a chat interface with an LLM called "AskLLM" that can:
\begin{itemize}
        \item Scrollable list of chat bubbles (user vs. LLM) with the newest chats at the bottom
        \item Input area containing a multiline text editor with a Submit button
\end{itemize}


\subsection{Non-Functional Requirements}

\subsubsection{Performance Requirements}
\begin{itemize}
  \item The system shall handle message delivery latency below 200 ms.
  \item The Kanban interface shall render task updates within 100 ms.
\end{itemize}

\subsubsection{Availability}
\begin{itemize}
  \item The system shall maintain 99.5\% uptime.
\end{itemize}

\subsubsection{Security}
\begin{itemize}
  \item All communication shall be encrypted using HTTPS.
  \item Authentication shall be handled via JWT tokens.
\end{itemize}

\subsubsection{Operating System and Environment}
\begin{itemize}
  \item This application is developed and deployed for Apple devices running iOS, iPadOS, and macOS, which require iOS 26, iPadOS 26, or macOS Sequoia 26 as per the latest Apple developer policies.
\end{itemize}

% \appendix
% \addcontentsline{toc}{section}{Appendix}
% \section*{Appendix}
% \section{References}

\end{document}
