\documentclass[11pt]{article}
\usepackage{graphicx} % To include graphics 
\usepackage[tbtags]{amsmath} % tbtags to make eq number at last line
\usepackage{amssymb}
%\usepackage{wrapfig} % For figures next to text
\usepackage{booktabs} % Better tables
% \usepackage{physics} 
\usepackage{cases} % For cases with separate labels
\usepackage{hyperref}% For cross-referencing
\hypersetup{colorlinks=true, linkcolor=blue, urlcolor=blue, citecolor=blue}
%\usepackage{pgfplots} % For drawing graphs
\usepackage{parskip} % To remove indentation and set paragraph spacing
\usepackage[margin=1in]{geometry}
%\usepackage{cancel} % To cancel out terms
\usepackage{listings} % To write code
\usepackage{xcolor} % To write color text/math by \textcolor{color}{text}
\usepackage{enumitem} % Better list environment
\usepackage{fancyhdr} % For custom headers and footers
% \usepackage{amsthm} % To write theorems and proofs
% \renewcommand\qedsymbol{$\blacksquare$}
% \usepackage{algorithm}
% \usepackage{algpseudocode}

%%%%%%%%%% Configure fancy headers %%%%%%%%%%
\pagestyle{fancy}

%%%%%%%%%% listings config (for writing code) %%%%%%%%%%
% \definecolor{codegreen}{rgb}{0,0.6,0}
% \definecolor{codegray}{rgb}{0.5,0.5,0.5}
% \definecolor{codepurple}{rgb}{0.58,0,0.82}
% \definecolor{backcolour}{rgb}{0.95,0.95,0.92}

% \lstdefinestyle{mystyle}{
%     frame=tb,
%     basicstyle=\ttfamily,
%     backgroundcolor=\color{backcolour},   
%     % commentstyle=\color{codegreen},
%     % keywordstyle=\color{magenta},
%     % stringstyle=\color{codepurple},
%     breakatwhitespace=false,         
%     breaklines=true,                 
%     captionpos=b,                    
%     keepspaces=true,                 
%     numbers=left,                    
%     numbersep=5pt,                  
%     showspaces=false,                
%     showstringspaces=false,
%     showtabs=false,                  
%     tabsize=4
% }
% \lstset{style=mystyle}

% Title
\title{CSCI3100 Project Requirements Specification}
\author{Group 17: Ng Ching Yin (1155175606), Lam Hoi Chun (1155192755), ...}
\date{\today}

\begin{document}
\maketitle

% Table of contents
{
    \hypersetup{linkcolor=black}
    \tableofcontents
}

\newpage

% Document Revision History
% Update this only before pull request, don't update every commit
\section{Document Revision History}
\begin{table}[h]
    \centering
    \caption{Document Revision History}
    \begin{tabular}{cccc}
        \toprule
        Version & Revised By & Revision Date & Comments \\
        \midrule
        nil & nil & nil & nil \\
        \bottomrule
    \end{tabular}
    \label{tab:docs_rev_hist}
\end{table}

% Introduction
\section{Introduction}

\subsection{Purpose}

This document defines the user requirements for the new software project management
tool for the Demon Slayers Corps. This document will serve as a basis for both
stakeholders and developers in subsequent development activities.

\subsection{Project Background}

In a realm where demon slayers fought valiantly against the forces of darkness, the boss of
the Demon Slayer Corps struggled with managing their software projects, leading to chaotic
battles and the tragic loss of many brave warriors. Recognizing the dire need for better
organization, Kei, the CSCI3100 lecturer, and his eager students decided to step in, determined
to create a robust software project management tool tailored specifically for the Demon Slayers.

With their combined skills in coding and project management, they envisioned a system that
would streamline communication, track progress, and allocate resources effectively, ultimately
empowering the slayers to focus on their true mission: vanquishing demons and protecting
humanity.

\subsection{Business Objectives}
The high-level business objectives for the project are to:
\begin{enumerate}
    \item reduce inefficiency in task and resource allocation, including
    \begin{itemize}%[leftmargin=*]
        \item[(a)] average time from task creation to assignment;
        \item[(b)] workload variance across team members;
    \end{itemize}
    \item track mission and development progress in real time;
    \item streamline communication among developers;
    \item provide useful tools to facilitate software development;
\end{enumerate}

\subsection{Project Scope}
The new project management tool aims to streamlines software team coordination by 
providing task tracking, scheduling, messaging, and a LLM chat interface to accelerate
development workflows.

The major functions to be provided by the tool are listed below:
\begin{enumerate}
    \item Kanban board for task management and progress tracking;
    \item Calendar to track and visualize deadlines;
    \item Messaging system to streamline communication among developers;
    \item LLM chat interface to speed up development progress;
\end{enumerate}

% Assumptions and Dependencies
\section{Assumptions and Dependencies}
\begin{table}[h]
    \centering
    \caption{Assumptions and Dependencies}
    \begin{tabular}{p{2cm} p{3.5cm} p{10cm}}
        \toprule
        Assumption No. 
        & Title 
        & Description \\
        
        \midrule
        A1 
        & Supported OS
        & All popular OS, including Windows, Linux, macOS, iOS, Android and iPadOS \\

        \midrule
        A2
        & Apple minimum OS version 
        & iOS 26, iPadOS 26, and macOS 26  \\
 
        \bottomrule
    \end{tabular}
    \label{tab:docs_assumptions_and_dependencies}
\end{table}

\newpage
% Requirements
\section{Requirements}

\subsection{Completing the Software Development Process}

\subsubsection{Requirements Specification}

\subsubsection{Design and Implementation}

\subsubsection{Testing}

\subsubsection{Delivery}

\subsection{Software Requirements}

\subsubsection{Global Database}

\subsubsection{User Interface}

\subsubsection{User Management}

\subsubsection{License Management}

\subsubsection{Application-Specific Functionalities Stated in the Software Requirements
Specification}

\subsubsection{Operating System and Environment}

This application is developed and deployed for Apple devices running iOS, iPadOS, and macOS, which require iOS 18, iPadOS 18, or macOS Sequoia 15 (or later), as per the latest Apple developer policies.


\subsubsection{Code}

\subsubsection{Hardware}

% Sections of "Documents", "Source Control", "Grouping", "Other Requirements" are not copied from the template as they are the instructions of this document or the project

\subsection{List of Requirements}
\begin{table}[h]
    \centering
    \caption{List of Requirements}
    \begin{tabular}{ccc}
        \toprule
        Requirement No. & Title & Description \\
        \midrule
        R1 & TBD & TBD \\
        R2 & TBD & TBD \\
        \bottomrule
    \end{tabular}
    \label{tab:docs_rev_hist}
\end{table}

\end{document}
