\section{System Architecture}

\subsection{Architectural Patterns}
\subsubsection{Model-View-ViewModel (MVVM)}
The application will follow the Model-View-ViewModel design pattern, where
View represents the graphical user interface (GUI), Model controls the business logic and data, and
ViewModel acts as a middleman to control the view updates and forward user requests to the model.

This pattern enables separation of business logic and user interface, enhances code reusability
and makes the application easier to test and maintain.

\subsubsection{Singleton}
For some simple UX components, for example \verb|ThemeManager| for themes and \verb|AudioManager| 
for sound effects, we simply use global singleton classes in order to ensure update to all views 
and the APIs are accessible to all components.

Due to time constraint, \verb|DevbanUserContainer| also adopts the Singleton pattern so that the user
information could be accessed globally. In the future, this should be modified to the dependency
injection pattern for better testability.

\subsection{Decomposition Description}
Figure \ref{fig:decomposition_description} shows a high level decomposition of the components in devban. 
Users interact with the system via a GUI, which forward the actions to the view models and models.
The view models will then update the views according to the action and the data from database.

Some external services are used to extend the system capabilities:
\begin{itemize}%[leftmargin=*]
    \item Firebase Authentication are used to provide secure user authentication;
    \item Firebase Firestore are used as an online realtime database;
    \item LLM Provider API are used to obtain access to advanced LLM models;
\end{itemize}
The details for each components will be discussed in the following sections.

\begin{figure}[h]
    \centering
    \includegraphics[width=0.8\textwidth]{assets/system_architecture/system_architecture.pdf}
    \caption{Components diagram for high level overview of devban application.}
    \label{fig:decomposition_description}
\end{figure}