\documentclass[11pt]{article}
\usepackage{graphicx} % To include graphics 
\usepackage[tbtags]{amsmath} % tbtags to make eq number at last line
\usepackage{amssymb}
%\usepackage{wrapfig} % For figures next to text
\usepackage{booktabs} % Better tables
% \usepackage{physics} 
\usepackage{cases} % For cases with separate labels
\usepackage{hyperref}% For cross-referencing
\hypersetup{colorlinks=true, linkcolor=blue, urlcolor=blue, citecolor=blue}
%\usepackage{pgfplots} % For drawing graphs
\usepackage{parskip} % To remove indentation and set paragraph spacing
\usepackage[margin=0.5in, tmargin=1in, bmargin=1in]{geometry}
%\usepackage{cancel} % To cancel out terms
\usepackage{listings} % To write code
\usepackage{xcolor} % To write color text/math by \textcolor{color}{text}
\usepackage{enumitem} % Better list environment
\usepackage{fancyhdr} % For custom headers and footers
% \usepackage{amsthm} % To write theorems and proofs
% \renewcommand\qedsymbol{$\blacksquare$}
% \usepackage{algorithm}
% \usepackage{algpseudocode}
\usepackage{float} % For floating figure
\usepackage{subcaption} % For subfigures
\usepackage[none]{hyphenat} % To prevent a word across two line
\hyphenpenalty=10000
\exhyphenpenalty=10000

%%%%%%%%%% Configure fancy headers %%%%%%%%%%
\pagestyle{fancy}

%%%%%%%%%% listings config (for writing code) %%%%%%%%%%
% \definecolor{codegreen}{rgb}{0,0.6,0}
% \definecolor{codegray}{rgb}{0.5,0.5,0.5}
% \definecolor{codepurple}{rgb}{0.58,0,0.82}
% \definecolor{backcolour}{rgb}{0.95,0.95,0.92}

% \lstdefinestyle{mystyle}{
%     frame=tb,
%     basicstyle=\ttfamily,
%     backgroundcolor=\color{backcolour},   
%     % commentstyle=\color{codegreen},
%     % keywordstyle=\color{magenta},
%     % stringstyle=\color{codepurple},
%     breakatwhitespace=false,         
%     breaklines=true,                 
%     captionpos=b,                    
%     keepspaces=true,                 
%     numbers=left,                    
%     numbersep=5pt,                  
%     showspaces=false,                
%     showstringspaces=false,
%     showtabs=false,                  
%     tabsize=4
% }
% \lstset{style=mystyle}

% Title
\title{CSCI3100 Project Release Notes and User Manual}
\author{
    \textbf{Group 17} \\[1em]
    \begin{tabular}{ll}
        Ng Ching Yin & (1155175606) \\
        Lam Hoi Chun & (1155192755) \\
        Lai Wing Fai & (1155177159) \\
        Zou Zhi Hong & (1155204947)
    \end{tabular}
}
\date{\today}

\begin{document}
\maketitle

% % Table of contents
% {
%     \hypersetup{linkcolor=black}
%     \tableofcontents
% }

% Document Revision History
% Update this only before pull request, don't update every commit
\section*{Document Revision History}
\begin{table}[h]
    \centering
    \caption{Document Revision History}
    \begin{tabular}{p{1.5cm} p{6cm} p{4cm} p{3cm}}
        \toprule
        Version & Revised By & Revision Date & Comments \\
        \midrule
        0.0.1 & Ng Ching Yin, Lam Hoi Chun & \today & Initial release \\
        \bottomrule
    \end{tabular}
    \label{tab:docs_rev_hist}
\end{table}

\section{Introduction}

\textbf{devban} is a collaborative task management application designed specifically
for development teams. It combines Kanban task
organization, calendar scheduling, team discussion and LLM chat interface
to help teams track and complete their work efficiently.

\begin{center}
    \fbox{%
        \begin{minipage}{0.9\linewidth}
            \textbf{Note}: Messages to testers are wrapped in boxes.
            These are to be removed in official user manual.
        \end{minipage}
    }
\end{center}

\section{Building and Setup}
\begin{center}
    \fbox{%
        \begin{minipage}{0.8\linewidth}
            Ideally the app should be distributed via App Store,
            but since this is just a school project it is not feasible to get it published.
        \end{minipage}
    }
\end{center}
Here is a step-by-step guide to build devban on your Mac (See Figure \ref{fig:Xcode}).

\begin{enumerate}
    \item Download and install XCode from macOS App Store.
    \item Ensure your device meets the system requirements (iOS / iPadOS / macOS 26.0 or later).
    \item Open Xcode and clone the repository from \url{https://github.com/alvinng4/devban}. Open devban.
    \item Go to signing and capabilities. Create a Xcode account and change the team to your account.
    \item Select \verb|My Mac (Mac Catalyst)| and build the application by \verb|Cmd + R| clicking the Run button.
\end{enumerate}
\begin{center}
    \fbox{%
        \begin{minipage}{0.8\linewidth}
            \textbf{Note:} Testers don't have to setup the database. The app comes with a public API key that connects to our database automatically.
        \end{minipage}
    }
\end{center}

\begin{figure}[h]
    \centering
    \begin{subfigure}[b]{0.48\textwidth}
        \centering
        \includegraphics[width=\textwidth]{Xcode1.png}
        \caption{Cloning and opening devban via XCode Interface}
        \label{fig:Xcode1}
    \end{subfigure}
    \hfill
    \begin{subfigure}[b]{0.48\textwidth}
        \centering
        \includegraphics[width=\textwidth]{Xcode2.png}
        \caption{XCode setup for running devban}
        \label{fig:Xcode2}
    \end{subfigure}
    \caption{Graphical guide for building devban on XCode}
    \label{fig:Xcode}
\end{figure}

\section{Create / Login Account}

\subsection{Account Creation}
To create a new account:
\begin{enumerate}[leftmargin=*]
    \item Launch the app and click \verb|Create an account| on the welcome screen.
    \item Choose one of the following methods:
    \begin{itemize}
        \item \textbf{Email/Password}: Enter your email address and create a secure password.
        \item \textbf{Google Sign-In}: Tap the Google Sign-In button for quick account creation.
    \end{itemize}
\end{enumerate}
If you chose email/password, a \textbf{verification email} will be sent to your email address.

\subsection{Login}
It is similar to account creation. Simply follow the graphical interface to login.

\subsection{Password Recovery}
If you forget your password, click \verb|Forgot password?| on the login screen and follow the instructions.

\section{Create / Join Team}
After successful login, you should see a team interface. You must create or join a team in order to use devban.

\subsection{Creating a Team}
\begin{center}
    \fbox{%
        \begin{minipage}{0.8\linewidth}
            For testers, we have prepared five license keys: 
            \begin{enumerate}
                \item \texttt{9KYhcpgognwICIlluhsV}
                \item \texttt{DFtafeHAVlBAl5LOsTlh}
                \item \texttt{JIR3iHsFJZVWDU9XvaAA}
                \item \texttt{OiyIbWhzboizdu6Wed3k}
                \item \texttt{mfJIBSye00mM1IIlK9RZ}
            \end{enumerate}
        \end{minipage}
    }
\end{center}
To create a new team, contact us via \url{csci3100group17@gmail.com}. 
After successful subscription to our service, your team would receive a \textbf{license key}.
Simply follow the instructions on the interface to create an account. 

\textbf{Note}: The account used will automatically become the team admin upon creation.

\subsection{Joining a Team}
To join an existing team, ask your admin to generate an invite code, which will expire in 7 days.
Follow the interface to enter the invite code.

\textbf{For admin}: To generate an invite code, simply navigate by \verb|Profile > Team > Generate Invite Codes|.

\begin{figure}[h]
    \centering
    \includegraphics[width=0.85\textwidth]{generating_invite_codes.png}
    \caption{Generating invite codes}
    \label{fig:generating_invite_codes}
\end{figure}

\section{Core Features}
\begin{figure}[h]
    \centering
    \includegraphics[width=0.9\textwidth]{home_screen.png}
    \caption{Home Tab of devban}
    \label{fig:home_tab}
\end{figure}

\subsection{Team Discussion Board (Home Tab)}
A place to communicate with team members in real time. Markdown formatting is supported.

\subsection{Kanban Board (Home Tab)}
Kanban board with task columns (To-do, In Progress, Done)

\subsubsection{Task Creation}
\begin{enumerate}
    \item Tap the \verb|+| button at the bottom of any column.
    \item Enter task title and description.
    \item Set difficulty level.
    \item Assign a deadline (Optional).
    \item Pin important tasks (Optional).
\end{enumerate}

\subsubsection{Task Management}
\begin{enumerate}
    \item Complete / uncomplete a task by clicking on the checkbox.
    \item User can drag the task between columns to change the completion status (on wide screen only).
    \item To edit and check details of a task, simply tap on any task item.
    \item To delete a task, go into detail view and click the rubbish bin icon at the top right corner.
\end{enumerate}

\subsection{Calendar}
Visual representation of kanban task deadlines. Dates with deadline tasks will be annotated by a dot.
\begin{enumerate}
    \item Simply click on a date to check all the tasks that have deadlines on that day.
    \item Tap the \verb|+| button to create a task with deadline at the selected date.
\end{enumerate}

\subsection{AskLLM}
LLM chat interface with local Apple Intelligence model (more models will be added in the future).
\begin{itemize}
    \item Support Markdown formatting
    \item Unlimited rate with Apple Intelligence model (However, this model is not supported on old devices)
\end{itemize}

\subsection{Gamification System}
A gamification system is included to boost the motivation for developers.
By completing tasks, they can earn experience (exp) and level up, where each level = 100 exp.
(See Table \ref{tab:gamification})
\begin{table}[H]
    \centering
    \caption{Experience against difficulty level for completing a task}
    \begin{tabular}{c|c|c|c|c|c}
        \toprule
            Difficulty & Very Easy & Easy & Normal & Hard & Very Hard  \\
        \midrule
            Experience & 5 & 10 & 20 & 40 & 80 \\
        \bottomrule
    \end{tabular}
    \label{tab:gamification}
\end{table}

\subsection{Customizing Your Experience}
We provide customizations to enhance user experience. Simply navigate via \verb|Profile > Settings| to
customize your settings:
\begin{itemize}
    \item \textbf{Theme}: Multiple color themes
    \item \textbf{Dark Mode}: Auto, Dark, or Light mode
    \item \textbf{Sound Effects}: Optional sound feedback
    \item \textbf{Haptic Feedback}: Optional haptics feedback
\end{itemize}

\section{Support}

For issues or suggestions, contact: \url{csci3100group17@gmail.com}.

When reporting issues, please include: device model, OS version, app version, steps to reproduce, screenshots, and error messages (if any).

\end{document}
