\documentclass[11pt]{article}
\usepackage{graphicx} % To include graphics 
\usepackage[tbtags]{amsmath} % tbtags to make eq number at last line
\usepackage{amssymb}
%\usepackage{wrapfig} % For figures next to text
\usepackage{booktabs} % Better tables
% \usepackage{physics} 
\usepackage{cases} % For cases with separate labels
\usepackage{hyperref}% For cross-referencing
\hypersetup{colorlinks=true, linkcolor=blue, urlcolor=blue, citecolor=blue}
%\usepackage{pgfplots} % For drawing graphs
\usepackage{parskip} % To remove indentation and set paragraph spacing
\usepackage[margin=1in]{geometry}
%\usepackage{cancel} % To cancel out terms
\usepackage{listings} % To write code
\usepackage{xcolor} % To write color text/math by \textcolor{color}{text}
\usepackage{enumitem} % Better list environment
\usepackage{fancyhdr} % For custom headers and footers
% \usepackage{amsthm} % To write theorems and proofs
% \renewcommand\qedsymbol{$\blacksquare$}
% \usepackage{algorithm}
% \usepackage{algpseudocode}

%%%%%%%%%% Configure fancy headers %%%%%%%%%%
\pagestyle{fancy}

%%%%%%%%%% listings config (for writing code) %%%%%%%%%%
\definecolor{codegreen}{rgb}{0,0.6,0}
\definecolor{codegray}{rgb}{0.5,0.5,0.5}
\definecolor{codepurple}{rgb}{0.58,0,0.82}
\definecolor{backcolour}{rgb}{0.95,0.95,0.92}

\lstdefinestyle{mystyle}{
    frame=tb,
    basicstyle=\ttfamily,
    backgroundcolor=\color{backcolour},   
    commentstyle=\color{codegreen},
    keywordstyle=\color{magenta},
    stringstyle=\color{codepurple},
    breakatwhitespace=false,         
    breaklines=true,                 
    captionpos=b,                    
    keepspaces=true,                 
    numbers=left,                    
    numbersep=5pt,                  
    showspaces=false,                 
    showstringspaces=false,
    showtabs=false,                  
    tabsize=4
}
\lstset{style=mystyle}

% Title
\title{CSCI3100 Project User Manual}
\author{
    \textbf{Group 17} \\[1em]
    \begin{tabular}{ll}
        Ng Ching Yin & (1155175606) \\
        Lam Hoi Chun & (1155192755) \\
        Lai Wing Fai & (1155177159) \\
        Zou Zhi Hong & (1155204947)
    \end{tabular}
}
\date{\today}

\begin{document}
\maketitle

% Table of contents
{
    \hypersetup{linkcolor=black}
    \tableofcontents
}

\newpage

% Document Revision History
\section{Document Revision History}
\begin{table}[h]
    \centering
    \caption{Document Revision History}
    \begin{tabular}{p{1.5cm} p{6cm} p{2cm} p{3cm}}
        \toprule
        Version & Revised By & Revision Date & Comments \\
        \midrule
        1.0.0 & Group 17 & \today & Initial user manual release \\
        \bottomrule
    \end{tabular}
    \label{tab:docs_rev_hist}
\end{table}

\section{Introduction}

\textbf{Devban} is a collaborative task management application designed for development teams. It combines Kanban-style task organization, calendar scheduling, AI-powered assistance, and gamification elements to help teams track and complete their work efficiently.

\subsection{Key Capabilities}

\begin{itemize}
    \item \textbf{Team Collaboration}: Create or join teams to work together on projects
    \item \textbf{Kanban Board}: Organize tasks in To-do, In Progress, and Done columns
    \item \textbf{Calendar Integration}: View and manage tasks with deadlines
    \item \textbf{AI Assistant}: Get help from an LLM-powered chat interface
    \item \textbf{Team Discussion}: Communicate with team members in real-time
    \item \textbf{Gamification}: Earn experience points and level up by completing tasks
\end{itemize}

% Main Interface
\begin{figure}[htbp]
    \centering
    \includegraphics[width=0.65\linewidth]{main_tab.png}
    \caption{Main Interface}
    \label{fig:placeholder}
\end{figure}

\section{System Requirements}

\subsection{Development Environment}

To build and develop the application, you will need:

\begin{itemize}
    \item \textbf{macOS}: macOS 14.0 (Sonoma) or later
    \item \textbf{Xcode}: Version 15.0 or later (recommended: 16.0+)
    \item \textbf{Swift}: Version 6.2
    \item \textbf{iOS Deployment Target}: iOS 17.0 or later (note: project file shows 26.0, which may need adjustment)
\end{itemize}

\subsection{Runtime Requirements}

To run the application on a device:

\begin{itemize}
    \item \textbf{iOS Device}: iPhone or iPad running iOS 17.0 or later
    \item \textbf{Internet Connection}: Required for authentication and data synchronization
    \item \textbf{Firebase Account}: Required for backend services
    \item \textbf{Google Account}: Optional, for Google Sign-In functionality
\end{itemize}

\subsection{Dependencies}

The application uses the following Swift Package Manager dependencies:

\begin{itemize}
    \item \textbf{Firebase iOS SDK} (v12.5.0+): Authentication, Firestore database, Analytics
    \item \textbf{Google Sign-In iOS} (v9.0.0+): Google authentication integration
    \item \textbf{swift-markdown-ui} (v2.4.1+): Markdown rendering for chat messages
    \item \textbf{Splash} (v0.16.0+): Syntax highlighting for code blocks
    \item \textbf{SwiftLintPlugins} (v0.62.1+): Code linting and formatting
\end{itemize}

\section{Building and Setup}

\subsection{Prerequisites}

\subsubsection{Install Xcode}

\begin{enumerate}
    \item Download Xcode from the Mac App Store or \href{https://developer.apple.com/xcode/}{Apple Developer}
    \item Ensure Command Line Tools are installed by running:
    \begin{lstlisting}[language=bash]
xcode-select --install
    \end{lstlisting}
\end{enumerate}

\subsubsection{Install SwiftFormat}

For code formatting, install SwiftFormat using Homebrew:

\begin{lstlisting}[language=bash]
brew install swiftformat
\end{lstlisting}

\subsubsection{Firebase Project Setup}

\begin{enumerate}
    \item Create a Firebase project at \href{https://console.firebase.google.com/}{Firebase Console}
    \item Enable Authentication:
    \begin{itemize}
        \item Go to Authentication → Sign-in method
        \item Enable Email/Password authentication
        \item Enable Google Sign-In provider
    \end{itemize}
    \item Create a Firestore database:
    \begin{itemize}
        \item Go to Firestore Database
        \item Create database in production mode (or test mode for development)
        \item Configure security rules appropriately
    \end{itemize}
    \item Download \texttt{GoogleService-Info.plist}:
    \begin{itemize}
        \item Go to Project Settings → General
        \item Download the \texttt{GoogleService-Info.plist} file
        \item Ensure it's placed in the \texttt{devban/} directory
        \item Add it to the Xcode project target
    \end{itemize}
\end{enumerate}

\subsubsection{Google Sign-In Configuration}

\begin{enumerate}
    \item Configure OAuth 2.0 credentials in \href{https://console.cloud.google.com/}{Google Cloud Console}
    \item Add the reversed client ID to \texttt{Info.plist} (already configured in the project)
    \item Ensure the bundle ID matches: \texttt{com.chingyinng.devban}
    \item Configure the OAuth consent screen if required
\end{enumerate}

\subsection{Building the Application}

\subsubsection{Clone the Repository}

\begin{lstlisting}[language=bash]
git clone <repository-url>
cd devban
\end{lstlisting}

\subsubsection{Open in Xcode}

\begin{lstlisting}[language=bash]
open devban.xcodeproj
\end{lstlisting}

\subsubsection{Resolve Dependencies}

\begin{itemize}
    \item Xcode will automatically resolve Swift Package Manager dependencies
    \item If not, go to: \textbf{File → Packages → Resolve Package Versions}
\end{itemize}

\subsubsection{Configure Signing}

\begin{enumerate}
    \item Select the \texttt{devban} target in Xcode
    \item Go to \textbf{Signing \& Capabilities}
    \item Select your development team
    \item Ensure the bundle identifier matches: \texttt{com.chingyinng.devban}
\end{enumerate}

\subsubsection{Verify Configuration}

\begin{itemize}
    \item Ensure \texttt{GoogleService-Info.plist} is included in the target
    \item Check that \texttt{Info.plist} contains the Google Sign-In URL scheme
    \item Verify entitlements are properly configured
\end{itemize}

\subsubsection{Build and Run}

\begin{enumerate}
    \item Select a simulator or connected device
    \item Press \texttt{Cmd + R} or click the Run button
    \item Wait for the build to complete and the app to launch
\end{enumerate}

\subsection{Code Formatting}

Before committing code, format it using SwiftFormat:

\begin{lstlisting}[language=bash]
swiftformat devban/
\end{lstlisting}

Fix any SwiftLint warnings:
\begin{itemize}
    \item Warnings are shown in Xcode's Issue Navigator
    \item Follow the project's SwiftLint configuration in \texttt{.swiftlint.yml}
\end{itemize}

\section{Getting Started}

\subsection{First Launch}

\subsubsection{Launch the Application}

When you first launch the application:
\begin{itemize}
    \item The app will initialize Firebase and check authentication status
    \item You'll see a loading screen briefly
    \item If not authenticated, you'll be taken to the sign-in screen
\end{itemize}

\subsubsection{Sign In or Sign Up}

\begin{itemize}
    \item If you don't have an account, tap ``Create an account''
    \item Enter your email and password
    \item Or use ``Sign in with Google'' for quick access
\end{itemize}

% Login-in Screen
\begin{figure}[htbp]
    \centering
    \includegraphics[width=0.55\linewidth]{login_screen.png}
    \caption{Login Screen}
\end{figure}


\subsubsection{Join or Create a Team}

After authentication, you must join or create a team:

\begin{itemize}
    \item \textbf{Join Team}: Enter an invite code provided by a team admin
    \item \textbf{Create Team}: Provide a team name and license ID (if required)
\end{itemize}

% Team Authentication Screen
\begin{figure}[htbp]
    \centering
    \includegraphics[width=0.55\linewidth]{team_tab.png}
    \caption{Team Authentication Screen}
\end{figure}

\subsubsection{Start Using the App}

Once in a team, you'll see the main interface with four tabs:
\begin{itemize}
    \item \textbf{Home}: Kanban board and team discussion
    \item \textbf{Calendar}: Task calendar view
    \item \textbf{AskLLM}: AI assistant chat
    \item \textbf{Profile}: User settings and team information
\end{itemize}

\section{Features Overview}

\subsection{Main Navigation}

The application uses a tab-based navigation system with four main sections:

\begin{enumerate}
    \item \textbf{Home Tab}
    \begin{itemize}
        \item Kanban board with task columns (To-do, In Progress, Done)
        \item Team discussion chat
        \item User gamification progress bar
    \end{itemize}

    % Home Tab
    \begin{figure}[htbp]
        \centering
        \includegraphics[width=0.55\linewidth]{main_tab.png}
        \caption{Home Tab}
        \label{fig:placeholder}
    \end{figure}

    \newpage
    
    \item \textbf{Calendar Tab}
    \begin{itemize}
        \item Monthly calendar view
        \item Task deadlines visualization
        \item Task list for selected dates
    \end{itemize}

    % Calendar Tab
    \begin{figure}[htbp]
        \centering
        \includegraphics[width=0.55\linewidth]{calendar_tab.png}
        \caption{Calendar Tab}
    \end{figure}
        
    \item \textbf{AskLLM Tab}
    \begin{itemize}
        \item AI-powered chat interface
        \item Code syntax highlighting
        \item Conversation history management
    \end{itemize}

    % AskLLM Tab
    \begin{figure}[htbp]
        \centering
        \includegraphics[width=0.55\linewidth]{AskLLM_tab.png}
        \caption{AskLLM Tab}
    \end{figure}

    \newpage
    
    \item \textbf{Profile Tab}
    \begin{itemize}
        \item User profile and level information
        \item Team details and management
        \item Application settings
        \item Account management
    \end{itemize}

    % Profile Tab
    \begin{figure}[htbp]
        \centering
        \includegraphics[width=0.55\linewidth]{profile_tab.png}
        \caption{Profile Tab}
    \end{figure}
\end{enumerate}





\section{Detailed Feature Guide}

\subsection{Authentication}

\subsubsection{Google Sign-In}

\begin{enumerate}
    \item Tap the ``Sign in with Google'' button
    \item Select your Google account
    \item Grant necessary permissions
    \item You'll be automatically signed in
\end{enumerate}

\newpage

\subsubsection{Email/Password Sign-In}

\begin{enumerate}
    \item Enter your email address
    \item Enter your password
    \item Tap ``Sign In''
    \item If you forget your password, tap ``Forget password?'' and follow the reset instructions
\end{enumerate}

% Forgot Password 
\begin{figure}[htbp]
    \centering
    \includegraphics[width=0.55\linewidth]{forgot_pw.png}
    \caption{Forgot Password}
\end{figure}

\subsubsection{Sign Up}

\begin{enumerate}
    \item From the sign-in screen, tap ``Create an account''
    \item Enter your email address
    \item Create a secure password
    \item Complete the sign-up process
    \item Verify your email if required
\end{enumerate}

\subsection{Team Management}

\subsubsection{Creating a Team}

\begin{enumerate}
    \item After signing in, you'll see the team authentication screen
    \item Select ``Create team'' tab
    \item Enter a team name
    \item Provide a license ID (if required by your organization)
    \item Tap ``Create''
    \item You'll automatically become the team admin
\end{enumerate}

% Create Team
\begin{figure}[htbp]
    \centering
    \includegraphics[width=0.55\linewidth]{create_team_tab.png}
    \caption{Create Team}
\end{figure}

\subsubsection{Joining a Team}

\begin{enumerate}
    \item Select ``Join team'' tab
    \item Enter the invite code provided by a team admin
    \item Tap ``Join''
    \item You'll be added as a team member
\end{enumerate}

% Join Team
\begin{figure}[htbp]
    \centering
    \includegraphics[width=0.55\linewidth]{team_tab.png}
    \caption{Join Team}
\end{figure}


\subsubsection{Team Features}

\begin{itemize}
    \item \textbf{View Team Info}: See team name, your role, and member count in Profile
    \item \textbf{Generate Invite Codes}: Team admins can generate invite codes (valid for 7 days)
    \item \textbf{Exit Team}: Leave your current team (available in Profile settings)
\end{itemize}

% Profile Tab

\subsection{Kanban Board (Home Tab)}

\subsubsection{Task Columns}

The Kanban board displays three columns:
\begin{itemize}
    \item \textbf{To-do}: Tasks that haven't been started
    \item \textbf{In Progress}: Tasks currently being worked on
    \item \textbf{Done}: Completed tasks
\end{itemize}

% Kanban

\subsubsection{Creating Tasks}

\begin{enumerate}
    \item Navigate to the Home tab
    \item Tap the ``+'' button in the desired column
    \item Fill in task details:
    \begin{itemize}
        \item \textbf{Title}: Task name (required)
        \item \textbf{Description}: Detailed task information (supports Markdown)
        \item \textbf{Difficulty}: Very Easy, Easy, Normal, Hard, or Very Hard
        \item \textbf{Progress}: 0-100\% completion slider
        \item \textbf{Deadline}: Optional date and time
        \item \textbf{Pin Task}: Keep task at top of list
    \end{itemize}
    \item Tap ``Save'' to create the task
\end{enumerate}

% New Task
\begin{figure}[h]
    \centering
    \includegraphics[width=0.55\linewidth]{new_task.png}
    \caption{Create New Task}
\end{figure}

\subsubsection{Managing Tasks}

\begin{itemize}
    \item \textbf{View Task}: Tap any task card to see full details
    \item \textbf{Edit Task}: Open task details and tap ``Edit''
    \item \textbf{Delete Task}: Swipe left on a task card and tap ``Delete''
    \item \textbf{Move Task}: Drag and drop tasks between columns (on wide screens)
    \item \textbf{Complete Task}: Mark as done to earn experience points
\end{itemize}

\subsubsection{Task Difficulty and Experience Points}

Experience points are awarded based on task difficulty:
\begin{itemize}
    \item \textbf{Very Easy}: 5 EXP
    \item \textbf{Easy}: 10 EXP
    \item \textbf{Normal}: 20 EXP
    \item \textbf{Hard}: 40 EXP
    \item \textbf{Very Hard}: 80 EXP
\end{itemize}

Experience points are awarded when you mark a task as completed. If you uncomplete a task, the EXP is deducted.

\subsubsection{Layout Adaptations}

The Home tab adapts to screen size:
\begin{itemize}
    \item \textbf{Wide (>900px)}: All columns visible side-by-side
    \item \textbf{Medium (600-900px)}: Discussion + tabbed task columns
    \item \textbf{Narrow (<600px)}: All sections in tabs
\end{itemize}

\subsection{Calendar View}

\subsubsection{Navigation}

\begin{itemize}
    \item \textbf{Previous/Next Month}: Use arrow buttons in the header
    \item \textbf{Today}: Tap ``Today'' to jump to the current date
    \item \textbf{Select Date}: Tap any date in the calendar grid
\end{itemize}

\subsubsection{Viewing Tasks}

\begin{itemize}
    \item Dates with tasks show visual indicators
    \item Tap a date to see all tasks with deadlines on that date
    \item Tap a task in the list to view details
\end{itemize}

\subsubsection{Creating Tasks with Deadlines}

\begin{enumerate}
    \item Navigate to Calendar tab
    \item Select a date (or use today)
    \item Tap the ``+'' button in the header
    \item Create the task (deadline is pre-filled with selected date)
    \item Save the task
\end{enumerate}

\subsection{AskLLM (AI Assistant)}

\subsubsection{Starting a Conversation}

\begin{enumerate}
    \item Navigate to AskLLM tab
    \item Type your question or request in the text editor
    \item Tap ``Submit'' or press Enter
    \item Wait for the AI response (streaming response shown in real-time)
\end{enumerate}

\subsubsection{Conversation Management}

\begin{itemize}
    \item \textbf{Stop Model}: Interrupt a generating response
    \item \textbf{Clear Context}: Clear conversation history while keeping the session
    \item \textbf{Restart Conversation}: Start a completely new conversation
    \item \textbf{Dismiss Keyboard}: Hide the on-screen keyboard
\end{itemize}


\subsubsection{Features}

\begin{itemize}
    \item Markdown rendering for formatted responses
    \item Code syntax highlighting
    \item Conversation history maintained during session
    \item Streaming responses for real-time feedback
\end{itemize}


\subsection{Team Discussion}

\subsubsection{Sending Messages}

\begin{enumerate}
    \item Navigate to Home tab
    \item Open the ``Discussion'' section
    \item Type your message in the text editor
    \item Tap ``Submit'' to send
    \item Messages are visible to all team members
\end{enumerate}

\subsubsection{Viewing Messages}

\begin{itemize}
    \item Messages appear in chronological order
    \item User names and timestamps are displayed
    \item Markdown formatting is supported
\end{itemize}

\subsection{Profile and Settings}

\subsubsection{User Profile}

\begin{itemize}
    \item \textbf{Display Name}: Your account name
    \item \textbf{Level}: Current level (based on experience points)
    \item \textbf{Experience}: Current EXP and progress to next level
    \item \textbf{Level Calculation}: Level = (EXP / 100) + 1
\end{itemize}

\subsubsection{Team Information}

\begin{itemize}
    \item \textbf{Team Name}: Your current team
    \item \textbf{Role}: Admin or Member
    \item \textbf{Member Count}: Number of team members
    \item \textbf{Generate Invite Codes}: Create new invite codes (Admin only)
    \item \textbf{Exit Team}: Leave the current team
\end{itemize}

\subsubsection{General Settings}

\paragraph{Dark Mode}
\begin{itemize}
    \item \textbf{Auto}: Follows system setting
    \item \textbf{Dark}: Always use dark mode
    \item \textbf{Light}: Always use light mode
\end{itemize}

\paragraph{Theme}
\begin{itemize}
    \item Choose from available color themes (Blue, Green, Orange, etc.)
    \item Theme affects button colors and accent colors throughout the app
\end{itemize}

\paragraph{Sound Effects}
\begin{itemize}
    \item Toggle sound effects on/off
    \item Plays sounds for certain actions (e.g., task completion)
\end{itemize}

\paragraph{Haptic Effects}
\begin{itemize}
    \item Toggle haptic feedback on/off
    \item Provides tactile feedback for interactions
\end{itemize}

% Settings Tab
\begin{figure}[h]
    \centering
    \includegraphics[width=0.75\linewidth]{settings_tab.png}
    \caption{Settings Tab}
\end{figure}

\subsubsection{Account Settings}

\paragraph{Logout}
\begin{enumerate}
    \item Navigate to Profile tab
    \item Expand ``Account'' settings
    \item Tap ``Logout''
    \item Confirm logout
    \item You'll be returned to the sign-in screen
\end{enumerate}

\paragraph{Delete Account}
\begin{enumerate}
    \item Navigate to Profile tab
    \item Expand ``Account'' settings
    \item Tap ``Delete account''
    \item Follow the confirmation steps
    \item \textbf{Warning}: This action is permanent and cannot be undone
\end{enumerate}

\section{Assumptions and Expected Behavior}

\subsection{Authentication Assumptions}

\begin{itemize}
    \item Users must have a valid email address or Google account
    \item Password reset emails are sent via Firebase Authentication
    \item Google Sign-In requires appropriate OAuth configuration
    \item Users must be authenticated to access any app features
\end{itemize}

\subsection{Team Assumptions}

\begin{itemize}
    \item \textbf{Every user must belong to a team} to use the application
    \item Users can only belong to one team at a time
    \item Team creation requires a license ID (may be optional depending on configuration)
    \item Invite codes expire after 7 days
    \item Invite codes can only be used once
    \item Team admins can generate unlimited invite codes
\end{itemize}

\subsection{Task Management Assumptions}

\begin{itemize}
    \item Tasks are team-scoped (visible to all team members)
    \item Tasks can have three statuses: To-do, In Progress, Completed
    \item Tasks can optionally have deadlines
    \item Tasks can be pinned to appear at the top of lists
    \item Progress is tracked as a percentage (0-100\%)
    \item Difficulty levels determine experience point rewards
    \item Completing a task awards EXP; uncompleting deducts EXP
\end{itemize}

\subsection{Data Synchronization Assumptions}

\begin{itemize}
    \item All data is stored in Firebase Firestore
    \item Changes sync in real-time across all team members
    \item Internet connection is required for all operations
    \item Offline functionality is limited (Firebase handles caching)
\end{itemize}

\subsection{UI/UX Assumptions}

\begin{itemize}
    \item The app adapts to different screen sizes (iPhone, iPad)
    \item Layout changes based on horizontal size class (compact vs. regular)
    \item Theme preferences are user-specific and synced
    \item Sound and haptic preferences are user-specific
    \item Color scheme can override system settings
\end{itemize}

\subsection{Gamification Assumptions}

\begin{itemize}
    \item Experience points are awarded only when tasks are marked as completed
    \item EXP is calculated based on task difficulty
    \item Level increases every 100 EXP
    \item EXP can be negative if tasks are uncompleted (but level won't go below 1)
    \item EXP is stored per user, not per team
\end{itemize}

\subsection{Calendar Assumptions}

\begin{itemize}
    \item Calendar shows tasks with deadlines
    \item Tasks without deadlines don't appear on the calendar
    \item Multiple tasks can have the same deadline
    \item Calendar navigation is month-based
\end{itemize}

\subsection{LLM Chat Assumptions}

\begin{itemize}
    \item AskLLM requires an active internet connection
    \item Responses are streamed in real-time
    \item Conversation context is maintained during the session
    \item Clearing context starts fresh but keeps the session
    \item Restarting conversation creates a new session
\end{itemize}

\subsection{Error Handling Assumptions}

\begin{itemize}
    \item Network errors are displayed to users
    \item Authentication errors show appropriate messages
    \item Invalid invite codes show error messages
    \item Task operations show errors if they fail
    \item The app attempts to handle errors gracefully
\end{itemize}

\section{Troubleshooting}

\subsection{Build Issues}

\paragraph{Dependencies not resolving}
\begin{itemize}
    \item \textbf{Solution}: Go to File → Packages → Reset Package Caches, then Resolve Package Versions
\end{itemize}

\paragraph{Code signing errors}
\begin{itemize}
    \item \textbf{Solution}: Ensure your Apple Developer account is configured in Xcode preferences
    \item Check that the bundle identifier matches your provisioning profile
\end{itemize}

\paragraph{Firebase not initializing}
\begin{itemize}
    \item \textbf{Solution}: Verify \texttt{GoogleService-Info.plist} is in the project and added to the target
    \item Check that Firebase is properly configured in the Firebase Console
\end{itemize}

\subsection{Runtime Issues}

\paragraph{Cannot sign in}
\begin{itemize}
    \item \textbf{Solution}: 
    \begin{itemize}
        \item Verify internet connection
        \item Check that Firebase Authentication is enabled
        \item Ensure email/password authentication is enabled in Firebase Console
        \item For Google Sign-In, verify OAuth configuration
    \end{itemize}
\end{itemize}

\paragraph{Cannot join team}
\begin{itemize}
    \item \textbf{Solution}:
    \begin{itemize}
        \item Verify the invite code is correct and not expired
        \item Check that the invite code hasn't been used already
        \item Ensure you're not already in a team
    \end{itemize}
\end{itemize}

\paragraph{Tasks not appearing}
\begin{itemize}
    \item \textbf{Solution}:
    \begin{itemize}
        \item Check internet connection
        \item Verify you're viewing the correct team's tasks
        \item Refresh the view by navigating away and back
        \item Check Firestore database rules
    \end{itemize}
\end{itemize}

\paragraph{AskLLM not responding}
\begin{itemize}
    \item \textbf{Solution}:
    \begin{itemize}
        \item Verify internet connection
        \item Check that the LLM API is properly configured
        \item Try restarting the conversation
        \item Check for error messages in the chat
    \end{itemize}
\end{itemize}

\paragraph{Calendar not showing tasks}
\begin{itemize}
    \item \textbf{Solution}:
    \begin{itemize}
        \item Ensure tasks have deadlines set
        \item Check that the selected date is correct
        \item Verify tasks are assigned to your team
    \end{itemize}
\end{itemize}

\subsection{Performance Issues}

\paragraph{App is slow}
\begin{itemize}
    \item \textbf{Solution}:
    \begin{itemize}
        \item Check internet connection speed
        \item Close and restart the app
        \item Check device storage space
        \item Update to the latest iOS version
    \end{itemize}
\end{itemize}

\paragraph{High data usage}
\begin{itemize}
    \item \textbf{Solution}:
    \begin{itemize}
        \item The app syncs data in real-time, which uses data
        \item Consider using Wi-Fi for large operations
        \item Firestore caches data locally to reduce usage
    \end{itemize}
\end{itemize}

\section{Contact and Support}

\subsection{Development Team}

\textbf{Group 17 - CSCI3100 Project (CUHK)}
\begin{itemize}
    \item Lai Wing Fai
    \item Lam Hoi Chun
    \item Ng Ching Yin
    \item Zou Zhi Hong
\end{itemize}

\subsection{Contact Information}

\begin{itemize}
    \item \textbf{Email}: \href{mailto:csci3100group17@gmail.com}{csci3100group17@gmail.com}
    \item \textbf{Project}: CSCI3100 at The Chinese University of Hong Kong
\end{itemize}

\subsection{Reporting Issues}

When reporting issues, please include:
\begin{itemize}
    \item Device model and iOS version
    \item App version (found in Profile → About)
    \item Steps to reproduce the issue
    \item Screenshots if applicable
    \item Error messages (if any)
\end{itemize}

\subsection{Code Contribution}

Before submitting code:
\begin{enumerate}
    \item Format code with SwiftFormat
    \item Fix all SwiftLint warnings
    \item Test on multiple device sizes
    \item Ensure Firebase rules are updated if needed
    \item Document significant changes
\end{enumerate}

\appendix
\addcontentsline{toc}{section}{Appendix}
\section*{Appendix}

\section{Keyboard Shortcuts}

\begin{itemize}
    \item \textbf{Submit Message}: Enter key (in AskLLM and Discussion)
    \item \textbf{Dismiss Keyboard}: Use the keyboard dismiss button in toolbars
\end{itemize}

\section{Supported Markdown}

The app supports Markdown in:
\begin{itemize}
    \item Task descriptions
    \item Chat messages (AskLLM and Discussion)
    \item Code blocks with syntax highlighting
\end{itemize}

\section{Data Structure}

\begin{itemize}
    \item \textbf{Users}: Stored in \texttt{users} collection
    \item \textbf{Teams}: Stored in \texttt{teams} collection
    \item \textbf{Tasks}: Stored in \texttt{tasks} collection
    \item \textbf{Invite Codes}: Stored in \texttt{team\_invite\_codes} collection
    \item \textbf{Licenses}: Stored in \texttt{licenses} collection
\end{itemize}

\section{Security Notes}

\begin{itemize}
    \item All authentication is handled by Firebase
    \item Data is encrypted in transit
    \item Firestore security rules should be configured appropriately
    \item User passwords are never stored locally
    \item Google Sign-In uses OAuth 2.0
\end{itemize}

\end{document}
